\makeatletter
\def\saveenum{\xdef\@savedenum{\the\c@enumi\relax}}
\def\resetenum{\global\c@enumi\@savedenum}
\makeatother

\chapter{Considerações Finais}
\label{ch:consideracoes}

Neste trabalho é apresentado o objetivo de realizar uma análise de um período grade (6 meses) de análise do tráfego, na Universidade do Estado de Santa Catarina. Após um estudo de trabalhos encontrados na área foi possível verificar que os períodos das análises registradas em períodos anteriores não são realizados em períodos grandes (maiores que um mes ou até mesmo dias), está trabalho foca em uma análise em um período maior de tempo (6 meses), com o objetivo de verificar o surgimento de alguns fatores/características que não foram observados devido ao período da análise destes outros trabalhos.

\section{O que foi feito}\label{sec:o_que_foi_feito}

No início deste período de estudo, que durou um total de três meses, foi realizado a formulação de um plano para o desenvolvimento deste trabalho, algumas das caracteríticas abordadas neste plano podem ser observadas como:

\begin{enumerate}[label=\textnormal{(\arabic*)}]
    \item Formulação do plano do TCC, especificação de algumas características para o início do trabalho;
    \item Revisão sobre DNS, \textit{honeypots} e DNSpot, para um melhor entendimento das principais características e funcionalidade da ferramenta que está sendo utilizada para a captura de informação;
    \item Revisão de trabalhos correlatos, identificação dos principais trabalhos relacionados com a pesquisa. Buscando analisar as análises utilizadas em outros trabalhos e o período que a pesquisa acabou ocorrendo. O principal fator analisado foi o período que as análises ocorreram;
    \item Adaptação no DNSpot para coletas de longo prazo, algumas modificações foram realizadas para que o DNSpot consiga lidar com o período de coleta, removendo algumas características que não torna-se essenciais para este estudo;
    \item Coleta de dados, o início da coleta de dados foi no dia 17/09/2016; %VERIFICAR ESSA DATA, LOG FILE TCC.....
    \item Definição das análises a serem realizadas, com um melhor entendimento e caracterização de alguns trabalhos na área, foi possível definir algumas características para a análise a ser realizada;
    \item Escrita da monografia da primeira parte (TCC-I).
\end{enumerate}


\section{Próximas etapas}\label{sec:proximas_etapas}


\begin{enumerate}[label=\textnormal{(\arabic*)}]
  \setcounter{enumi}{7}
    \item Análise dos resultados obtidos, com uma análise a longo prazo será possível observar algumas características e comportamentos não vistos em uma análise realizada em um período pequeno (um mes ou até mesmo dias);
    \item Escrita da monografia da segunda parte.
    \label{itm:1}
\end{enumerate}

\section{Cronograma}
\label{cro}

O cronograma proposto para a primeira etapa \ref{sec:o_que_foi_feito}, pode ser observador na tabela \ref{sec:consideracoes_cronograma}.

%%%% então acabei colocando ... mas acho q vou tirar HAHAH

\vspace{0.5cm}
{\tiny
\noindent \begin{tabular}{|c||c|c|c|c|c|c|c|c|c|c|c|c||c|c|c|c|c|c|c|c|c|c|c|c|}
  \hline
  \multirow{2}{*}{\textbf{\small{Etapas}}} & \multicolumn{12}{|c||}{\textbf{\small{2016}}} & \multicolumn{12}{|c|}{\textbf{\small{2017}}} \\
  \cline{2-25}
   & \textbf{J} & \textbf{F} & \textbf{M} & \textbf{A} & \textbf{M} & \textbf{J} & \textbf{J} & \textbf{A} & \textbf{S} & \textbf{O} & \textbf{N} & \textbf{D} & \textbf{J} & \textbf{F} & \textbf{M} & \textbf{A} & \textbf{M} & \textbf{J} & \textbf{J} & \textbf{A} & \textbf{S} & \textbf{O} & \textbf{N} & \textbf{D} \\
  \hline %\hline
  %\textbf{\small{1}} & & & & & & & x & x & & & & & & &  & & & & & & & & & \\
  \hline
  \textbf{\small{1}} & & & & & & & & x & x & & & & & & &  & & & & & & & & \\
  \hline
  \textbf{\small{2}} & & & & & & & &  & x & x & & & & & & & & & & & & & & \\
  \hline
  \textbf{\small{3}} & & & & & & & & & x & & & & & & & & & & & & & & & \\
  \hline
  \textbf{\small{4}} & & & & & & & & & & x & x & x & x & x & x & x & x
  & x & & & & & & \\
  \hline
  \textbf{\small{5}} & & & & & & & & & & x & x & & & & & & & & & & & & & \\
  \hline
  \textbf{\small{6}}  &  & & & & & & & & & & & & & x & x & x & x & x & & & & & & \\
  \hline
  \textbf{\small{7}} & & & & & & & & & & & & & & & & x & x & x & & & & & & \\
  \hline
%  \textbf{\small{9}} & & & & & & & & & & & & & & & & & x &  & & & & & & \\
%  \hline
%  \textbf{\small{10}} & & & & & & & & & & & & & & & & & x & x & x & & & & & \\
%  \hline  
%  \textbf{$\cdots$} & & & & & & & & & & & & & & & & & & & & & & & & \\
%  \hline
\end{tabular}
}

REFAZER ESSA TABELAAAAAA
\label{sec:consideracoes_cronograma}
