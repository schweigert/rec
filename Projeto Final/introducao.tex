\chapter{Introdução}
\label{ch:intro}

O DNS (\textit{Domain Name System}) \cite{rfc1034} é um sistema
distribuído de resolução de nomes que desempenha um papel fundamental
na Internet. Sua principal funcionalidade é traduzir nomes de domínio
mais facilmente memorizáveis (como \url{www.udesc.br}) em endereços IP
numéricos (como 200.19.105.194), que são usados pelos protocolos
subjacentes de rede para localizar e identificar nós na Internet.

Em vista de sua ampla utilização, o DNS também é tanto um alvo quanto
um vetor de ataques. As principais ameaças envolvendo o DNS são
resumidas por \cite{Conrad:2012:tidssr}, que as divide em duas
classes, aquelas em que o DNS é o alvo e aquelas que são oportunizadas
pelo DNS\@. A classe de ameaças ao DNS inclui:

\begin{itemize}

\item Negação de serviço: impedir o acesso de usuários ao DNS, com isso
  prejudicando ou mesmo bloqueando o seu acesso à Internet;

\item Corrupção de dados: modificar dados publicados no DNS de forma
  não autorizada, o que pode, por exemplo, levar usuários a acessar
  sites ilegítimos (como páginas falsas de bancos ou comércio
  eletrônico);

\item Exposição de informação: revelar informações sobre o
  comportamento dos usuários, como histórico de sites web acessados \cite{cert3}.
\end{itemize}

O DNS também pode ser usado como um veículo de ataques. A classe de
ameaças oportunizadas pelo DNS abrange:

\begin{itemize}
\item Ataques de amplificação: servidores DNS mal configurados podem ser
  usados para realizar ataques de negação de serviço contra terceiros \cite{cert3};
  %[CERT.br, Servidores DNS recursivos abertos]

\item \textit{Fast flux} DNS: servidores usados para propósitos
  nefastos, como propagação de software malicioso ou controle remoto
  de \textit{botnets}, podem ter diversos endereços IP distintos
  associados. Uma fração desses endereços são associados a um nome DNS
  específico e trocados com alta frequência, de modo a dificultar a
  localização dos servidores e a identificação dos seus responsáveis,
  e até mesmo balancear carga entre servidores
  \cite{Salusky:2007:ffsn};

\item Exfiltração de dados: como o tráfego DNS geralmente não é
  barrado ou modificado por \textit{firewalls}, ele é usado com
  frequência para transmitir dados sensíveis (capturados no curso
  de uma invasão) sem que isso seja percebido pelos mecanismos de
  defesa da rede.
\end{itemize}  
  
O maior ataque DDoS foi registrado em setembro de 2016, que conseguiu alcançar picos de 1 Tbps de tráfego.

O serviço de hospedagem OVH (\textit{Hosted service provider company}) na França foi a vítima, dos ataques que conseguiu alcançar a marca de 1 Tbps por segundo, os ataques acabaram utilizando \textit{Smart Devices} para a realização dos ataques. Dispositivos inteligentes faziam parte da \textit{botnet} que foi utilizada para a realização dos ataques. 

Octave Klaba, o fundador e CTO (\textit{chief technology officer}) da OVH, apresentou os diversos ataques que foram realizados, e o volume de tráfego que excedia 100 Gbps e atingiam picos de 799 Gbps. O ataque foi realizado por mais de 152000 dispositivos da IoT (\textit{Internet of Things}), que incluíam câmeras e gravadores de vídeo comprometidos \cite{report:ddos}.


\iffalse 
%- O que é o DNS, qual o seu papel na Internet 
Apesar da grande popularização dos meios de pesquisa e viabilização de diferentes tarefas, como transações bancarias. A internet com mais de 1 bilão de usuários, ainda possui fatores desconhecidos por seus usuários \cite{radware}. 
%% i  try,

Um serviço de DNS (Domain Name System) é responsável pela tradução de um \textit{host name}, ao receber uma URL (\textit{Uniform Resource Locators}), será responsável pela sua tradução para um endereço \textit{ip}.


%- DNS sujeito a ataques (Conrad 2012) -- citar exemplos (sem detalhes)
Ataques que buscam tirar proveito de serviços de DNS são comuns, por serem serviços abertos e acessíveis para qualquer usuário, e ainda por serem essenciais para a tradução de \textit{host name}, acabam sendo explorados para o envio de mensagens para os seus alvos. 

A tática utilizada pelos atacantes segue a lógica de enviar uma mensagem para o DNS com o endereço de retorno sendo o da máquina alvo, e o DNS ao invés de responder o cliente inicial responde ao alvo com a mensagem. Um ataque vai utilizar uma quantidade em uma escala muito superior para causar maiores danos.
 
A maioria dos ataques observados por US-CERT, são consultas realizadas do tipo \textit{any}, que acaba retornando todas as informações conhecidas de uma zona em uma única solicitação \cite{uscert}. Com a utilização de \textit{botnet} é possível criar uma quantidade elevada de tráfico com pouco gasto.
%https://www.us-cert.gov/ncas/alerts/TA13-088A

Ataques DDoS (Distributed Denial-of-Service attack) buscam tirar proveito de DNS, para a sua realização.

Um ataque DDoS tem o objetivo de exaurir recursos, para causar indisponibilidade. Os sistemas que geralmente acabam sofrendo com estes ataques são servidores web, como por exemplo, servidores de jogos ou lojas, que devido ao número elevado de mensagens recebidas, acaba ocorrendo uma sobrecarga do sistema.

Por consequência ao ataque, acaba-se gerando problemas com excesso de arquivos logs, e problemas com backups, também é possível considerar uma consequência a grande perdas financeiras.

Observando a relação de ataques DDoS reportados para a CERT.br, no ano de 2014 foi alcançado um total de 1.047.031  ataques, é possível observar um crescimento de 197\% de 2013 para 2014. Apesar de haver uma queda de 30\% em 2015, os valores ainda estão altos em relação a anos anteriores. Um fator importante a ser notado é que, os incidentes apresentados foram relatados, precisamos considerar incidentes menores ou incidentes em geral que não foram relatados para CERT.br \cite{cert1}.
%http://www.cert.br/stats/incidentes/

No ano de 2015 a empresa \textit{arbornetworks} \cite{Arbor}, que é um provedor líder em soluções de DDoS, registrou na Ásia o maior ataque da história em relação a volume de dados, com um total de 334Gbps, no decorrer do ano de 2015 também foram identificados outros 25 ataques com um volume de dados maior que 100Gbps.
%http://br.arbornetworks.com/arbor-networks-registra-os-maiores-ataques-ddos-no-relatorio-q1-2015-ddos/
\fi

%% - DNSpot - o que é, principais resultados do TCC do Felipe 

Para observar o comportamento de atacantes contra servidores DNS, foi
desenvolvido o DNSpot \cite{Longo:2015:tcc}, um \textit{honeypot} DNS com
o propósito de monitorar e registrar o tráfego enviado a um serviço de
DNS recursivo aberto. \textit{Honeypots} são recursos computacionais
de segurança, cujo objetivo é serem sondados, atacados ou
comprometidos em um ambiente controlado
\cite{Steding-Jessen:2007:ulihs}.

O DNSpot foi implantado na rede da UDESC durante 49 dias, entre
09/09/2015 e 28/10/2015. Nesse período, o \textit{honeypot} processou
mais de 4 milhões de consultas DNS, mais de 99\% das quais
relacionadas a ataques distribuídos de negação de serviço (DDoS,
\textit{Distributed Denial of Service}). A análise dos dados coletados
revelou a existência de nomes DNS projetados para maximizar a
amplificação de tráfego nesse tipo de ataque. Conforme mostrado na
Tabela~\ref{table:resolvidos1}, nove dos 10 nomes ou registros de
recursos (RRs, \textit{resource records}) que apareceram com maior
frequência não podem mais ser aproveitados em ataques DDoS, seja
porque não estão mais ativos ou porque agora geram respostas
consideravelmente menores; uma nova verificação dos dados foi realizada em
19/08/2016, cerca de 10 meses após o fim do estudo original. Um outro
fato observado no estudo original foi o desaparecimento de domínios
usados em ataques DDoS; isso foi constatado especificamente para o
domínio l3x.ru, que aparece na Tabela~\ref{table:resolvidos1}.

\begin{table}[h!]
    \centering
    \begin{tabular}{|l|c|r|r|}
    \hline
                           &     &
                           \multicolumn{2}{c|}{\textbf{Resposta
                               (bytes)}}\\ \cline{3-4}
    \textbf{RR}& \textbf{Ativo?} & \textbf{2015} & \textbf{2016} \\\hline
     hehehe.ru. ANY        & sim & 3850 & 221\\ \hline
     mototrazit.ru. ANY    & não & 3853 & -- \\ \hline
     vp47.ru. ANY          & sim & 3959 & 151\\\hline
     l3x.ru. A             & não & 3875 & --\\\hline
     . ANY                 & sim & 1503 & 1790\\\hline
     gransy.com. ANY       & sim & 3591 & 594\\\hline
     vp47.ru A             & sim & 3892 & 91\\\hline
     lifemotodrive.ru. ANY & não & 3969 & --\\\hline
     nhl.msk.su. ANY       & sim & 3965 & 341\\\hline
     oi69.ru. A            & sim & 3637 & 91\\\hline
    \end{tabular}
        \caption{Situação em 19/08/2016 dos 10 RRs observados com maior
          frequência por Longo (2015)}
    \label{table:resolvidos1}
\end{table}

O objetivo deste trabalho de conclusão de curso é realizar uma coleta de
dados com o DNSpot durante um período significativamente mais longo
que o estudo original. Isso não apenas permitirá comparar dados
recentes com os dados originais, mas principalmente analisar a
evolução da atividade maliciosa ao longo de vários meses, buscando
respostas para questões como:

\begin{itemize}
\item Com que frequência aparecem nomes e domínios anômalos, i.e.,
  projetados para ataques DDoS?
\item Por quanto tempo esses domínios permanecem válidos e são usados
  em ataques?
\item Existe alguma característica sazonal no tipo ou no volume de
  ataques?
\item É possível correlacionar tráfego de ataques DDoS com fatores
  externos, como questões geopolíticas ou econômicas?
  
\item Quais as principais características de um ataque DDoS para um  serviços de jogos, existe alguma diferença?

\end{itemize}

Visando um período de coleta entre oito e nove meses, que vai contribuir com resultados que não
conseguiram ser observados no primeiro estudo, devido ao tempo.


\section{Objetivo}
\label{sec:objetivo}

\textbf{Objetivo geral}: Fazer uma análise da evolução temporal de dados coletados pelo DNSpot.


\section{Objetivos Específicos}
\label{sec:objetivos_especificos}

\noindent\textbf{Objetivos específicos}: 
Segue uma lista dos principais objetivos a serem realizados no estudo.
\begin{itemize}
    \item Realizar uma revisão bibliográfica abrangendo segurança do DNS,
  \textit{honeypots} e trabalhos relacionados;
    \item Operacionalizar armazenamento de longo prazo no DNSpot;
    \item Fazer uma coleta de longa duração;
    \item Comparar os resultados novos com os anteriores;
    \item Analisar a evolução temporal dos dados observados.
\end{itemize}


\section{Metodologia}
\label{sec:solucao}

Este trabalho de conclusão de curso consiste em uma pesquisa aplicada, tendo como principais métodos a pesquisa bibliográfica e o estudo de caso. Primeiramente, serão realizados um estudo sobre o DNSpot e uma revisão bibliográfica sobre aspectos de segurança do DNS e \textit{honeypots}. Nesta fase será avaliada a necessidade de adaptações no DNSpot para coleta de dados de longo prazo. Atualmente, os dados coletados pelo DNSpot são armazenados em um banco de dados, o qual é rotacionado manualmente quando se torna muito grande. Para uma coleta de longo prazo, possivelmente será necessário automatizar o procedimento de rotação da base de dados.

\iffalse
Para a realização da análise evolucional dos dados, será necessário realizar uma pesquisa aplicada buscando um aprofundamento nos conceitos e funcionalidades do \iffalse Primeiramente, serão realizados um estudo sobre o\fi DNSpot e uma revisão
bibliográfica sobre aspectos de segurança do DNS e
\textit{honeypots}. Nesta fase será avaliada a necessidade de
adaptações no DNSpot para coleta de dados de longo prazo. 
\fi 

Após, será iniciada a coleta de dados para a realização da análise, juntamente com o estudo. 

Por fim será realizado a comparação com os dados originais
\cite{Longo:2015:tcc} e observando como eles evoluem ao longo do
tempo. Serão investigadas tendências de mais longo prazo, 
como tempo de vida dos nomes usados em ataques DDoS e com que
frequência surgem novos nomes, por exemplo.



\section{Estrutura do Trabalho}
\label{sec:escopo}

Este trabalho está dividido em cinco capítulos. O Capítulo 1 é uma introdução sobre o reconhecimento de expressões matemáticas, as possíveis aplicações, os tipos de reconhecimento e as etapas do reconhecimento. No Capítulo 2 são apresentados os fundamentos básicos de processamento de imagens, aprendizado de máquina e análise estrutural. O Capítulo 3 é uma revisão dos métodos utilizados na literatura para cada uma das etapas do reconhecimento de expressões matemáticas. O Capítulo 4 apresenta a proposta de uma aplicação \textit{web} que utiliza um reconhecedor de expressões matemáticas e mostra quais métodos serão implementados em cada uma das etapas. O Capítulo 5 são as considerações finais deste trabalho, o que foi estudado, o que foi proposto, o que foi feito até o momento e qual serão as próximas etapas.
