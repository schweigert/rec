\chapter{Trabalhos Relacionados}
\label{ch:relacionados}

%%%% COLOCAR ALGO AKI
Os trabalhos encontrados visam observar ataques DDoS em servidores raiz, ou a análise do comportamento de algumas características na rede. Existe diferenças nos períodos e tipos de análise realizadas.
 
Está seção busca apresentar algumas destas características e demonstrar algumas relações e diferenças com o trabalho apresentado. Dentre as diversas áreas de estudo sobre o DNS, como monitoramento, análise e detecção de anomalias



\section{Conceitualização}

\textbf{David Conrad} descreve primeiramente em seu trabalho \cite{Conrad:2012:tidssr}, conceitos para o entendimento do DNS e sua funcionalidade, ainda demonstrando uma conceitualização histórica. O trabalho demonstra a necessidade dos serviços do DNS para o funcionamento da \textit{Internet}, juntamente com todas as ameaças que este serviço possui.


\section{Análises}

\textbf{Roberto Perdisci} busca apresentar em seu trabalho \cite{DBLP:conf/acsac/PerdisciCDL09}


\section{Comparativo}

Para melhor entender as relações entre cada trabalho apresentando na seção \ref{ch:relacionados}, foi criado a tabela \ref{tab:all} onde é apresentando as principais características de cada trabalho.



%%%falar 4 artigos de análise dos ataques (pq a importância e oq cada análise focava fazer)

%%%terminar falando do tempo das análises para juntar com a proposta
    
%%%falar das análises que foram utilizados (relação das análise com relação ao tempo de coleta de dados) (subsection nova)

%%%falar qual foi o foco voltado dos trabalhos, a diferença para o meu trabalho (consigo falar do tempo) (período de análise dos dados principalmente, (subsection nova))


