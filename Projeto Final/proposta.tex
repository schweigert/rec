\chapter{Proposta}
\label{ch:proposta}

O objetivo deste trabalho é desenvolver um estudo sobre a evolução temporal de dados coletados pelo DNSpot. Será levado em consideração as informações coletadas em um estudo anterior \cite{Longo:2015:tcc}.%citar o felipe.

O estudo observou interações de usuários potencialmente maliciosos com servidores DNS recursivos, observando a atitude tomada pelos usuários. A ferramenta foi implantada na rede da Universidade do Estado de Santa Catarina (UDESC), por um período de 49 dias.

Com as informações apresentadas pelo estudo \cite{Longo:2015:tcc}, será possível observar o surgimento de algumas características que não foram possíveis ser analisadas em um período de (49 dias). Possibilitando entender o comportamento de certos ataques não observados em outros estudos devido ao período que a análise foi proposta.

Levando em consideração os trabalhos apresentados na seção \ref{ch:relacionados}, é possível destacar os períodos de observação das redes, muitas das vezes por não serem desenvolvidos em períodos grandes de tempo, não é possível a realização de certas análise. Este trabalho busca apresentar uma análise em um período grande de tempo (6 meses), para a realização de uma análise evolucionária em relação aos dados observados.

Para efetuar a análise algumas modificações foram realizadas no DNSpot, focando em modificar algumas características do sistema, garantindo um melhor funcionamento em um período maior de tempo. Como relatado no trabalho  \cite{Longo:2015:tcc}, foram encontrados problemas devido ao crescimento do banco de dados, o que é um problema ao se realizar uma análise maior como a deste trabalho. 

A solução aborda foi a remoção de alguns campos que eram salvas no banco de dados, com isto o crescimento do banco foi controlado, mas ainda é possível observar que o banco vai apresentar um crescimento levando em consideração o período de análise. %citar o outro cara q eu n tenho o trabalho....

\subsection{Implantação}

%%%falar qual máquina está rodando
%%%Quando começou e até quando pretende ficar rodando a análise

Para a realização da coleta de dados, uma máquina foi disponibilizada pela UDESC. A máquina foi implantada na rede interna da universidade, onde está realizando a coleta de dados, que iniciaram no dia 17/09/2016. %OLHAR ESSA DATA...

A rede utiliza endereços reservados, por este motivo é necessário a utilização do NAT (Network Address Translation) para realizar o redirecionamento do tráfego para porta 53/UDP. O NAT é um método utilizado para remapear espaços de endereços IPs, realizando a modificação no cabeçalho dos pacotes no momento em que estes se encontram no dispositvo de roteamento \cite{Naugle:1998}. %cite o NAT

A configuração do \textit{hardware} e \textit{software} da máquina utilizada, segue:

\begin{itemize}
    \item Sistema Operacional: OpenBSD 5.7 i386;
    \item Processador: Intel Core2 Duo CPU E6550 @ 2.33GHz (``GenuineIntel" 686-class);
    \item Memória RAM: 1 GB;
    \item Servidor DNS recursivo: Unbound, versão 1.5.2;
    \item Python, versão 3.4.2;
    \item DNSLib, versão 0.9.4;
    \item SQLIte3, versão 3.8.6.
\end{itemize}


A máquina está localiza em um dos laboratórios da UDESC, onde realiza a coleta de dados 24/7, o serviço é verificado todos os dias para a garantia do seu funcionamento.

Durante o período que o sistema já esta executando as coletas, ocorreram algumas interrupções do serviço devido a queda de energia e interrupção do \textit{firewall} que da acesso a \textit{Internet} (consequencia da queda de energia).

\subsection{Tabelas no SQLite}

%http://stackoverflow.com/questions/784173/what-are-the-performance-characteristics-of-sqlite-with-very-large-database-file
%http://stackoverflow.com/questions/14451624/will-sqlite-performance-degrade-if-the-database-size-is-greater-than-2-gigabytes
%https://www.sqlite.org/limits.html


\subsection{Proposta de testes}
\label{sec:proposta_testes}

%%%falar as análises (discutir as análises dos artigos)

Levando em consideração os trabalhos apresentados na seção \ref{ch:relacionados}, e o trabalho realizado pelo \cite{Longo:2015:tcc}, foi determinar uma proposta de testes para este trabalho. A proposta leva em consideração algumas características observadas em trabalhos anteriores, e algumas propostas em comum, para realização de um comparativo entre os trabalhos.

Com os diferentes \textbf{períodos de monitoramento}, será possível realizar uma análise quanto a quantidade de informações capturadas em períodos diferentes. E realizar uma comparação entre a coleta realizada em um período inferior de tempo do \textit{Longo:2015:tcc}.


Com o total de \textbf{Transações} será possível realizar uma contagem do número de usuários com intenção maliciosa, já que é necessário a descoberta do endereço, como se trata de um servidor DNS não anunciado. Também será possível verificar o número de requisições que não foram respondidas e as transações ignoradas.

O \textbf{Volume de dados em bytes} ajudara para a descoberta dos tamanhos das consultas mais comuns juntamente com o tamanho das consultas e respostas solicitadas.

\textbf{Clientes IP}, para apresentar as localizações geográficas, juntamente com os endereços que realizaram o maior número de consultas. \textbf{Domínios} para o número de transações. Distribuição empírica de requisições por \textbf{RR} e número de transações.

\textbf{Ataques DoS} visando o tamanho de consulta e resposta, quantidade de ataques recebido.

\textbf{Anomalias} podem ser detectadas, levando em consideração os resultados já encontrados nos trabalhos anteriores \textit{Longo:2015:tcc}.

\textbf{Desaparecimento de domínios}, devido ao período estendido de análise será possível observar o surgimento e desaparecimento de muitos domínios, com os resultados é esperado conseguir um melhor entendimento deste comportamento devido ao período.



PRECISO REVISAR ISSO AKI