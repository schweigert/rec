\appendix
\iffalse
\chapter{Apêndice: \textit{Datasets}}

%\section{\textit{Datasets}}
%\label{sec:revisao_datasets}

A seguir são listadas \textit{datasets} publicadas e que podem ser utilizadas para pesquisas. Para cada uma são detalhados o tipo (\textit{online} ou \textit{offline}), manuscrito ou impresso, o formato da informação, aonde encontrar os dados e a descrição.

\begin{itemize}
	\item detexify-data \cite{kirsch2010detexify}
	\subitem \textbf{Tipo}: \textit{Online}/manuscrito.
	\subitem \textbf{Formato}: Formato YAML $(x, y, time)$. Pontos espaçados com dados temporais.
	\subitem \textbf{Lincença}: ODbL (ODC \textit{Open Database License}).
	\subitem \textbf{Tamanho Aproximado}: $136042$ símbolos até janeiro de 2015.
	\subitem \textit{\textbf{Website}}: \url{https://github.com/kirel/detexify-data}
	\subitem \textbf{Descrição}: Os símbolos são dígitos, alfabeto latino minúsculo e maiúsculo, letras gregas e outros símbolos matemáticos.
	
	\item HWRT \cite{thoma2015line}
	\subitem \textbf{Tipo}: \textit{Online}/manuscrito.
	\subitem \textbf{Formato}: Formato YAML $(x, y, timestamp)$. Pontos espaçados com dados temporais.
	\subitem \textbf{Lincença}: ODbL (ODC \textit{Open Database License}).
	\subitem \textbf{Tamanho Aproximado}: $151158$ símbolos até janeiro de 2015.
	\subitem \textit{\textbf{Website}}: \url{http://www.martin-thoma.de/write-math/data}
	\subitem \textbf{Descrição}: Esta \textit{dataset} é composta por $90\%$ da \textit{detexify-data}. Os símbolos são dígitos, alfabeto latino minúsculo e maiúsculo, letras gregas e outros símbolos matemáticos.
	
	\item CROHME
	\subitem \textbf{Tipo}: \textit{Online}/manuscrito.
	\subitem \textbf{Formato}: Formato InkML. XML modificado que contém as expressões matemáticas em grupos de traços. Cada traço é composto por um conjunto de pontos. Cada ponto tem as coordenadas e o \textit{timestamp}.
	\subitem \textbf{Lincença}: \textit{Creative Commons license} para uso acadêmico. Proibido uso comercial.
	\subitem \textbf{Tamanho Aproximado}: ?
	\subitem \textit{\textbf{Website}}: \url{http://www.isical.ac.in/~crohme/CROHME_data.html}
	\subitem \textbf{Descrição}: \textit{Dataset} popular usada na competição CROHME reconhecimento de expressões matemáticas. Os dados são símbolos e expressões matemáticas inteiras.
	
	\item MfrDB \cite{stria2012mfrdb}
	\subitem \textbf{Tipo}: \textit{Online}/manuscrito.
	\subitem \textbf{Formato}: XML composto por InkML and MathML.
	\subitem \textbf{Lincença}: Gratuita para uso não comercial.
	\subitem \textbf{Tamanho Aproximado}: 2018 fórmulas matemáticas até janeiro de 2016.
	\subitem \textit{\textbf{Website}}: \url{http://mfr.felk.cvut.cz/Database_download.html}
	\subitem \textbf{Descrição}: 2018 fórmulas matemáticas escritas por 232 participantes por interface gráfica via \textit{mouse}, tela \textit{touch} ou \textit{stylus pen}.
	
%	\item MathBrush \cite{maclean2011grammar}
%	\subitem \textbf{Tipo}:
%	\subitem \textbf{Formato}: 
%	\subitem \textbf{Lincença}: 
%	\subitem \textbf{Tamanho Aproximado}: ?
%	\subitem \textit{\textbf{Website}}: \url{}
%	\subitem \textbf{Descrição}:
	
	\item HAMEX \cite{quiniou2011hamex}
	\subitem \textbf{Tipo}: \textit{Online}/manuscrito.
	\subitem \textbf{Formato}: InkML.
	\subitem \textbf{Lincença}: Não divulgada. Porém, Se entrar em contato eles disponibilizam.
	\subitem \textbf{Tamanho Aproximado}: 4350 expressões matemáticas.
	\subitem \textit{\textbf{Website}}: \url{http://ivc.univ-nantes.fr/en/databases/HAMEX}
	\subitem \textbf{Descrição}: Expressões matemáticas manuscritas e faladas (áudio).
	
	\item MNIST \cite{lecun1998gradient}
	\subitem \textbf{Tipo}: \textit{Offline}/manuscrito
	\subitem \textbf{Formato}: Formato IDX. Armazena vetores e matrizes com valores numéricos que representam a imagem de um dígito.
	\subitem \textbf{Lincença}: \textit{Creative Commons Attribution-Share Alike} 3.0.
	\subitem \textbf{Tamanho Aproximado}: 60 mil dígitos e mais 10 mil para o conjunto de testes.
	\subitem \textit{\textbf{Website}}: \url{http://yann.lecun.com/exdb/mnist}
	\subitem \textbf{Descrição}: \textit{Dataset} muito popular de dígitos $[0,9]$ manuscritos.
	
%	\item InftyCDB-1 \cite{suzuki2005ground}
%	\subitem \textbf{Tipo}: \textit{Offline}/printed.
%	\subitem \textbf{Formato}: ?
%	\subitem \textbf{Lincença}: ..
%	\subitem \textbf{Tamanho Aproximado}: ?
%	\subitem \textit{\textbf{Website}}: \url{http://www.inftyproject.org/en/database.html}
%	\subitem \textbf{Descrição}: Símbolos e fórmulas matemáticas impressas totalizando $21056$ expressões matemáticas.
	
	\item UW-III \cite{phillips1998methodologies}
	\subitem \textbf{Tipo}: \textit{Offline}/impresso.
	\subitem \textbf{Formato}: formato XFIG.
	\subitem \textbf{Lincença}: Proprietária.
	\subitem \textbf{Tamanho Aproximado}: 25 páginas impressas de fórmulas matemáticas.
	\subitem \textit{\textbf{Website}}: \url{http://isis-data.science.uva.nl/events/dlia/datasets/uwash3.html}
	\subitem \textbf{Descrição}: Essa base de dados é paga e custa USD 100.
	
	\item \cite{chajri2016handwritten}
	\subitem \textbf{Tipo}: \textit{Offline}/manuscrito.
	\subitem \textbf{Formato}: Imagens
	\subitem \textbf{Lincença}: ?
	\subitem \textbf{Tamanho Aproximado}: 10379 símbolos.
	\subitem \textit{\textbf{Website}}: \url{http://www.ncbi.nlm.nih.gov/pmc/articles/PMC4786757/}
	\subitem \textbf{Descrição}: $10379$ símbolos do alfabeto latino, árabe e símbolos matemáticos.
	
\end{itemize}


%\chapter{Apêndice: Ferramentas }

%aaa
%\section{Ferramentas relacionadas}
%\label{sec:revisao_ferramentas_relacionadas}

\fi