\titulo{ANÁLISE DO GNS3 COMO FERRAMENTA AUXILIAR AO ENSINO DO PROTOCOLO HTTP POR MEIO DA COMUNICAÇÃO ENTRE REDES SIMULADAS}
\autor{Christopher Renkavieski \\ Lucas Machado Gutierrez \\ Marlon Henry Schweigert}
\nome{Christopher\\Lucas\\ Marlon}
\ultimonome{Renkavieski \\ Gutierrez \\ Schweigert}

\bacharelado \curso{Ciência da Computação}
\ano{2016}
\data {\today}
\cidade{Joinville}

\instituicao{Universidade do Estado de Santa Catarina}
\sigla{UDESC}
\unidadeacademica{Centro de Ciências Tecnológicas}

\orientador{Prof\textordmasculine  Charles Christian Miers}
\examinadorum{Prof\textordmasculine  Charles Christian Miers}
\examinadordois{Prof\textordmasculine  Guilherme Piegas Koslovski}

\ttorientador{$ $}
\ttexaminadorum{$ $}
\ttexaminadordois{$ $}

\newpage
\pagestyle{empty}

\maketitle

%\begin{dedicatoria}
%\noindent
%\end{dedicatoria}
%\noindent
%\newpage
%\begin{epigrafe}
%\noindent
%"xxxxxxx"
%--alguem
%\end{epigrafe}

%\agradecimento{Agradecimentos}
%caneta



%\resumo{Resumo}

%Do momento de sua invenção até hoje, a Internet foi responsável por mudanças radicais e profundas na sociedade, permitindo comunicação e acesso a informação a taxas nunca antes experimentadas [ref: livro kurose e ross].
%A definição mais básica de internet é uma rede de redes. Já a Internet, com “I” maiúsculo, é uma rede de redes específica, que hoje conecta computadores, dispositivos móveis e outros por todo o planeta, tendo como base os protocolos de transmissão TCP/IP, e sobre a qual está edificada a World Wide Web (WWW).

%Para que a conexão entre as diversas redes da Internet seja possível, um dos conceitos fundamentais é o de roteamento, conectando os principais serviços desejados a todos os usuários desta enorme rede.

%[Continua]

%\noindent \textbf{Palavras-chave:} \textit{GNC3}, \textit{Simulação de Redes}, \textit{HTTP}, \textit{Análise de Redes Simuladas}.

%\resumo{Abstract}

%From the moment of its invention until today, the Internet was responsible for radical and profound changes in society, allowing communication and access to information at rates never before experienced [ref: book kurose and ross].
%The most basic definition of the internet is a network of networks. The Internet, with a capital "I", is a specific network of networks, which today connects computers, mobile devices and others across the globe, based on TCP / IP transmission protocols, and on which World is built Wide Web (WWW).

%In order for the connection between different Internet networks to be possible, one of the fundamental concepts is routing, connecting the main desired services to all users of this huge network.

%[Resumo Traduzido]

%\noindent \textbf{Palavras-chave:} \textit{GNS3}, \textit{Simulated Networks}, \textit{HTTP}, \textit{Simulated Network Analysis}.


\tableofcontents
\listoffigures
\listoftables
\newpage
\chapter*{Lista de Abreviaturas\hfill} \addcontentsline{toc}{chapter}{Lista de Abreviaturas}
\listofsymbols

\newpage
\pagestyle{myheadings}
