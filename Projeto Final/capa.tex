\titulo{ANÁLISE DO GNS3 COMO FERRAMENTA AUXILIAR AO ENSINO DO PROTOCOLO HTTP POR MEIO DA COMUNICAÇÃO ENTRE REDES SIMULADAS}
\autor{Christopher Renkavieski \\ Lucas Machado Gutierrez \\ Marlon Henry Schweigert}
\nome{Christopher\\Lucas\\ Marlon}
\ultimonome{Renkavieski \\ Gutierrez \\ Schweigert}

\bacharelado \curso{Ciência da Computação}
\ano{2016}
\data {\today}
\cidade{Joinville}

\instituicao{Universidade do Estado de Santa Catarina}
\sigla{UDESC}
\unidadeacademica{Centro de Ciências Tecnológicas}

\orientador{Prof\textordmasculine  Charles Christian Miers}
\examinadorum{Prof\textordmasculine  Charles Christian Miers}
\examinadordois{Prof\textordmasculine  Guilherme Piegas Koslovski}

\ttorientador{$ $}
\ttexaminadorum{$ $}
\ttexaminadordois{$ $}

\newpage
\pagestyle{empty}

\maketitle

%\begin{dedicatoria}
%\noindent
%\end{dedicatoria}
%\noindent
%\newpage
%\begin{epigrafe}
%\noindent
%"xxxxxxx"
%--alguem
%\end{epigrafe}

%\agradecimento{Agradecimentos}
%caneta

\resumo{Resumo}
Devido ao grande crescimento e popularização da Internet nos últimos anos, ataques como DDoS estão cada vez mais frequentes, sendo difícil encontrar usuários que ao realizar algum serviço online não tenha sofrido com um ataque. O monitoramento do tráfego de um serviço de DNS recursivo, pode ser realizado para a descoberta de ataques DDoS. Uma ferramenta com estas características é o DNSpot, sendo responsável por registrar o tráfego de envio a um serviço de DNS recursivo aberto. Este trabalho de conclusão de curso busca realizar uma nova pesquisa sobre o DNSpot, realizando uma coleta de dados em um período entre oito e nove meses, realizando uma análise de tendências, com o tempo de vida dos nomes usados em ataques DDoS e com que frequência surgem novos nomes.

\noindent \textbf{Palavras-chave:} \textit{Domain Name Systeam (DNS)}, \textit{Segurança em redes}, \textit{DNSpot}.

\resumo{Abstract}

Due to the great growth and popularization of the Internet in recent years, as DDoS attacks are becoming more frequent, and difficult to find users to perform some online service has not suffered an attack. Monitoring traffic recursive DNS service can be performed for the discovery of DDoS attacks. A tool with these characteristics is the DNSpot, being responsible for recording sending traffic to an open recursive DNS service. This course conclusion work tries to make a new search on DNSpot, performing a data collection in a period between eight and nine months conducting an analysis of trends, with the lifetime of the names used in DDoS attacks and how often come new names.

\noindent \textbf{Palavras-chave:} \textit{Offline Recognition}, \textit{Handwriting Recognition}, \textit{Mathematical Expression}.

\tableofcontents
\listoffigures
\listoftables
\newpage
\chapter*{Lista de Abreviaturas\hfill} \addcontentsline{toc}{chapter}{Lista de Abreviaturas}
\listofsymbols

\newpage
\pagestyle{myheadings}
