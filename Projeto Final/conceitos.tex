\chapter{Conceitos}
\label{ch:intro}

\section{Redes de Computadores}
\label{ch:redes}
Uma rede de computadores é, em sua definição mais simples, um conjunto de computadores que trocam informações entre si. Para que essa troca de informações seja possível, são necessárias ao menos duas coisas: protocolos de comunicação e um mensageiro.

Um protocolo de comunicação é uma regra – ou um conjunto de regras – para que ambas as partes envolvidas em uma troca de mensagens possam entender uma à outra. Um exemplo de protocolo de comunicação é um idioma: em geral, duas pessoas somente se entendem se estiverem falando a mesma língua. Em redes de computadores tem-se, por exemplo, o conjunto de protocolos TCP/IP, que permite a comunicação entre diferentes computadores em uma rede.

Já o mensageiro é o responsável por levar a mensagem de uma parte à outra. Pode ser um carteiro para levar cartas, ondas mecânicas para transmitir som, ou um roteador para trocar pacotes de dados entre computadores.

As aplicações para redes de computadores são inúmeras. Em uma empresa, uma rede pode ser utilizada para que múltiplos empregados possam trabalhar num mesmo projeto simultaneamente, ou permitir a troca de mensagens entre diferentes pontos da empresa rapidamente. Um conjunto de computadores também pode compartilhar recursos para a execução de tarefas computacionalmente custosas, formando o que é conhecido como um cluster. Uma rede de computadores também pode ser utilizada para a transmissão de notícias, comunicação de voz e vídeo, armazenamento de informações, entre muitos outros.

Porém, possivelmente a aplicação mais popular e revolucionária para redes de computadores é a Internet. Uma internet é uma rede de redes de computadores, e a Internet, com “I” maiúsculo, é uma rede de redes específica, que comunica computadores e outros dispositivos por todo o planeta, e sobre a qual está edificada a World Wide Web (WWW).

Redes de computadores estão estabelecidas sobre uma série de convenções, protocolos e dispositivos físicos que, se fossem todos definidos, forneceriam material suficiente para um livro. No contexto deste trabalho, será importante definir os conceitos de roteador, switch e host.

\subsection{Roteadores e Switches}
\label{ch:roteadores e switches}

Na analogia mencionada, um switch \cite{radware} é como o mensageiro na comunicação entre computadores. Ele se conecta aos computadores de uma rede, e nele estão configurados os protocolos que permitem a comunicação entre esses computadores.

Já um roteador exerce o mesmo papel de um switch, porém, em vez de realizar a comunicação entre computadores, ele permite a comunicação entre outros switches. Na prática, isso faz com que um roteador intermedeie a comunicação entre diferentes redes de computadores, permitindo a existência de redes de redes, incluindo a Internet.

\section{Wire Shark}

\section{Virtual Box}

\section{GNS3}
GNS3 (\textit{Graphical Network Simulator 3}) é uma ferramenta OpenSource (Sob licensa MIT) para simulação, testes e análise de redes de computadores. O seu grande destaque aos demais encontra-se pela qualidade oferecida em um software totalmente gratuíto.
\subsection{Histórico}
GNS3 é um software que já vem com uma grande história no mercado.
...

\subsection{Licença}

\subsection{Características}
